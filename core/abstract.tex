%%%%%%%%%%%%%%%%%%%%%%%%%%%%%%%%%%%%%%%%%
% Latex Nice Article
% List of acronyms
%
% Author(s):
% Marwan Abdellah (abdelah.marwan@gmail.com)
%
% License:
% MIT License 
%
%%%%%%%%%%%%%%%%%%%%%%%%%%%%%%%%%%%%%%%%%

\begin{abstract}
% Abstract goes here.
\lipsum[1]
\end{abstract}
The \ac{DRR} presents a vital role in medical imaging procedures and radiotherapy applications. They allow the continuous monitoring of patient positioning during image guided therapies using multi-dimensional image registration. Conventional generation of \acs{DRR}s using spatial domain algorithms such as ray casting is associated with computational complexity of $\mathcal{O}(N^3)$. Fourier slice theorem is an alternative approach for generating the \acs{DRR}s in the \emph{k}-space with reduced time complexity. In this project, the students are expected to build a heterogeneous computing framework for \acs{DRR} generation.

% Keywords 
\paragraph*{Keywords}{\textit{Digitally reconstructed radiography, X-ray volume rendering, Fourier slice theorem, heterogeneous computing, \acs{GPU}, \acs{OpenCL}, \acs{OpenGL}}}

% Add a new page after the abstract. 
\newpage 